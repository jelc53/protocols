\documentclass[12pt,letterpaper,twoside]{article}

\usepackage{cme212}
\usepackage{xcolor}

\begin{document}

{\centering \textbf{Project 2 Short Answer Questions\\ Silvia Gong, Julian Cooper\\}}
\vspace*{-8pt}\noindent\rule{\linewidth}{1pt}

\paragraph{Question 1: } In our implementation, Alice and Bob increment their
Diffie-Hellman ratchets every time they exchange messages. Could the protocol
be modified to have them increment the DH ratchets once every ten messages 
without compromising confidentiality against an eavesdropper?

{\color{purple} Semantic security here means that an adversary eavesdropping
cannot distinguish between random and ciphertext produced by our algorithm. 
Whether this property is provided by our implementation depends on the security 
properties Galois Counter Mode which is our underlying encryption algorithm.

Since GCM is semantically secure, we can argue that our algorithm is also 
semantically secure. Moreover, if we reduce the frequency of DH Ratchet steps,
our algorithm will remain semantically secure since we are still performing the
symmetric ratchet which provides GCM with new message keys each time.

However, by reducing the frequency of our DH ratchets, we do compromise Forward
Secrecy since an attacker that steals our root key will be able to decrypt upto
the last 10 messages. }


\paragraph{Question 2: } What if they never update their DH keys at all?
Please explain the security consequences of this change with regards to
Forward Secrecy and Break-in Recovery.

{\color{purple} 
\textbf{Forward Secrecy: } refers to an attacker not being 
able to read past messages given knowledge of the current root and chain keys. 
If we do not perform the DH ratchet, the attacker would be able to read all 
past communications if they possess the root key output from the last DH 
ratchet step!

However, if the attacker only is able to steal a chain key, then our algorithm 
will still provide Forward Secrecy since we update these for every message sent 
via our one-way HMAC function (Symmetric Chain KDF).

\textbf{Break-in Recovery: } refers to our users being able to recover from 
an attack where the adversary steals one party's key. Since inputs for the 
sending and receiving chains are constant, the symmetric ratchet itself does 
not provide break-in recovery. If we do not update the DH keys at all, then 
we cannot recover from an attack and therefore would not provide Break-in 
Recovery. }


\paragraph{Question 3: } Consider the following conversation between Alice 
and Bob, protected via Double Ratchet Algorithm according to the spec:

\begin{verbatim}
A: Hey Bob, can you send me the locker combo?
A: I need to get my laptop
B: Sure, it's 1234!
A: Great, thanks! I used it and deleted the previous message.
B: Did it work?
\end{verbatim}

What is the length of the longest sending chain used by Alice? By Bob? 
Please explain.

{\color{purple} Longest sending chain used by Alice is length 2. This is 
occurs when Alice sends 2 messages without response from Bob. Since Bob 
did not respond in between, Alice completes a second symmetric ratchet 
to update the chain and message keys without needing to complete a DH 
ratchet to update the root key. }

{\color{purple} Longest sending chain used by Bob is length 1. This is 
because Bob never sends more than one message in sequence without Alice
responding. Therefore, Bob needs to perform a DH ratchet and creates and new 
sending chain every time he sends a message. }

\paragraph{Question 4: } Unfortunately, in the situation above, Mallory 
has been monitoring their communications and finally managed to compromise 
Alice's phone and steal all her keys just before she sent her third message.
Mallory will be unable to determine the locker combination. State and describe 
the relevant security property and justify why double ratchet provides this
property.

{\color{purple} Forward Secrecy. This property states that compromising long
term keys or current session key must not compromise past communications.

The double ratchet provides this property: output keys from the past appear 
random to an adversary who learns the KDF key at some point in time. This 
happens because each ratchet step uses a KDF which acts like a one-way 
PRF (deterministic and output looks random) so long as we include 
sufficient entropy. }


\paragraph{Question 5: } The method of government surveillance is deeply
flawed. Why might it not be as effective as intended? 

{\color{purple} We interpreted this question to be about citizens' ability to 
evade government eavesdropping. Idea: what is the user (citizen) simply did not 
follow the protocol and used the wrong government public key. Without updates
to the protocol, the government would likely not even know this had happened
unless it tried to decrypt every message!}

What are the major risks involved with this method?

{\color{purple} The major risk we saw was around the government becoming a single 
source of failure. For example, if the government was hacked, and their secret key was
stolen, the attacker could access and read all communciations that use the messanging
service. }


\paragraph{Question 6: } The SubtleCrypto library is able to generate signatures
in various ways, including both ECDSA and RSA keys. For both the ECDSA and RSA-based
signature technqiues, please compare:

\begin{enumerate}
    \item Which keys take longer to generate (timing {\color{blue} SubtleCrypto.generateKey})?
    {\color{purple} RSA-based techniques (eg. RSA-PSS) are much slower
    for generating keys: 300ms vs 2ms }

    \item Which signature takes longer to generate (timing {\color{blue} SubtleCrypto.sign})? 
    {\color{purple} RSA-based signatures take longer to generate: 5ms vs 1ms }

    \item Which signature is longer in length (length of output of {\color{blue} SubtleCrypto.sign})?
    {\color{purple} RSA signature byte length is longer: 512 vs 96 bytes}

    \item Which signature takes longer to verify (timing {\color{blue} SubtleCrypto.verify})?
    {\color{purple} ECDSA takes longer to verify: 0.26ms vs 0.74ms}

\end{enumerate}


\end{document}

