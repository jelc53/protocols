\documentclass[12pt,letterpaper,twoside]{article}

\usepackage{cme212}

\begin{document}

{\centering \textbf{Project 1 Short Answer Questions\\ Silvia Gong, Julian Cooper\\}}
\vspace*{-8pt}\noindent\rule{\linewidth}{1pt}

\paragraph{Question 1: } Briefly describe your method for preventing the
adversary from learning information about the lenths of the passwords stored in
your password manager.

We will want to padd our password strings to ensure they are of equal length
before encrypted with AES-GCM and stored. The encrypt() and decrypt() methods
from the subtleCrypto library handle padding (and unpadding) for us.


\paragraph{Question 2: } Briefly describe your method for preventing swap
attacks. Provide an argument for why the attack is prevented in your scheme.

We defend against swap and rollback attacks in our scheme by checking that the
SHA-256 hash of our json dump has not change between dump and load steps. To do
this, we need to be able to store the hash of our json dump in a trusted
location so we have a source of truth to check against.

In addition, we explicitly handle swap attacks by encrypting name || password,
not just password, when populating out key-value store in set(name, value).
This then allows us to check whether the first part of the decrypted "value"
matches name when we call get(name). This second line of defense is important
because we might not always be guaranteed a trusted storage medium as required
above.

Why does this work? Imagine our adversary switches the values of "service1"
and "service2". Then when we call get("service1") to retrieve its password, we
find that the first part of the decrypted "value" equals "service2". Since this
does not match "service1", we throw and exception "Potential swap attack!". If
no swap had occured, we would pass this check and return the decrypted password
string usual to the caller.


\paragraph{Question 3: } In our proposed defense against rollback attack, we
assume that we can store the SHA-256 hash in a trusted location beyond the reach
of an adversary. Is it necessary to assume that such a trusted location exists,
in order to defend against rollback attacks?

Yes, having a trusted storage location beyond reach of an adversary is required
in order to defend against rollback attacks. This is because rollback attacks
reproduce legitimate domain names and passwords (i.e. signed with the right
secret keys) and so key verification will not surface an attack.

\paragraph{Question 4: } Explain how you would do the look up if you had to use
a randomized MAC instead of HMAC. Is there a performance penalty involved, and if
so, what?

In the case of a randomized MAC, there may be many valid tags $t_1, t_2, ...$
for a given message $m$. Assuming the randomized MAC is secure, it must still
be difficult for an adversary to produce a new valid tag $t\prime$.

For key-value store look up, we no are no longer able to recompute tag of the
name and directly lookup key-value pair in O(1). Instead we will need to loop
through all n elements in the key-value store dictionary, and for each run the
randomized MAC verify function to query whether we are at the right key-value
pair. This look up algorithm will cost O(n * [time to verify]).


\paragraph{Question 5: } In our specification, we leak the number of records in
the password manager. Describe an approach to reduce the information leaked
about the number of records.

Instead of storing our key-value pairs in a "flat" dictionary, we could think
about storing we could think about using a binary tree.

\begin{itemize}

    \item Each node of the tree will contain pointers to its child nodes, until we reach
        the leaf nodes which will contain nullptrs for child nodes and data packet with
        key-value pairs of our key-value store.

    \item Since it is unlikely we have the exact number of key-value pairs as
        leaves of a sufficiently large binary tree (would need to be a power of
        2), we will then pad the remainder of the leaves with fake data of same
        size as our real key-value pair data packets.

    \item The height of a binary tree is ceil(log(n+1)) where n is the number of
        nodes (= 2*leaves-1). We assume the adversary can see the size of the
        tree and therefore know its height and number of leaves. However, they
        will not know how many of leaves represent real vs fake key-value pairs.

    \item Finally, there could be anywhere from ceil(log(n)) (number of
        leaves at previous level of tree) to ceil(log(n+1)) (number of leaves at
        current level of tree) real key-value pairs. This range is equal to
        floor(log(n)).

\end{itemize}

\paragraph{Question 6: } What is a way we can add multi-user support for
specific sites to our password manager system without compromising security for
other sites that these users may wish to store passwords of? That is, if Alice
and Bob wish to access one stored password (say for nytimes) that either of them
can get and update, without allowing the other to access their passwords for
other websites.

We first use Diffie-Helman key exchange to generate a shared secret between
Alice and Bob. We will call this the sharedMasterKey which Alice and Bob each
then store separately in their own respective trusted locations.

With our sharedMasterKey, both Alice and Bob can use this to encrypt and decrypt
any shared sites (domain names, e.g. nytimes). This need not compromise security
for other sites that both users wish to maintain single-user access to because
they can just continue to use their respective existing master keys for these.

\end{document}

